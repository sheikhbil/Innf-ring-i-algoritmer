\documentclass{article}

\usepackage[utf8]{inputenc}
\usepackage{amsmath}
\usepackage{enumitem}

\title{Assignment 1 - Stable Matching}
\author{Bilal Sheikh (Bilalas)}
\date{\today}

\begin{document}

\maketitle

\section*{Task 1: Explaining Stable Matching}

\textbf{Question:} Explain, in your own words, what it means for a matching to be stable.

\textbf{Answer:} 
A matching is considered \textit{stable} if there are no two elements (say, $x$ from one set and $y$ from another set) that would prefer each other over their current partners. In other words, for all pairs $(x, y)$, if $x$ is matched with $y'$ and $y$ is matched with $x'$, neither $x$ prefers $y$ over $y'$, nor $y$ prefers $x$ over $x'$. This ensures that no pair has an incentive to deviate from the given matching.

\section*{Task 2: Stable Matching Problem}

\textbf{Question:} Prove or disprove the statement about a stable matching when $x$ and $y$ are each other's top choices.

\textbf{Answer:} 
Consider an instance where $x$ and $y$ are ranked first on each other's preference list. The statement implies that $(x, y)$ must be part of every stable matching. However, this is not necessarily true. [Continue with the proof or disproof, depending on your analysis.]

\section*{Task 3: College Admissions with Ties}

\textbf{Question:} Does there always exist a perfect matching with the given notion of stability? Justify your answer.

\textbf{Answer:} 
In college admissions with ties, a stable matching might not always exist under the provided conditions. [Provide justification and any relevant mathematical or logical arguments here.]

\section*{Conclusion}

This assignment covers fundamental concepts in stable matching, including theoretical analysis and practical application in college admissions. The proofs provided support the understanding of stability in different contexts.

\end{document}
